\begin{frame}
\frametitle{Kernelizáció: 2. szabály}

Ha egy csúcs nincs benne egyetlen háromszögben sem, akkor töröljük.
\begin{itemize}
\item \uncover<2->{A v nincs benne egyetlen körben sem.}
\item \uncover<3->{Az Y $\rightarrow$ v, illetve v $\rightarrow$ X élekre nincs szükség a feedback arc setben.}
\item \uncover<4->{Az Y $\rightarrow$ X élek nem fognak megfordulni.}
\end{itemize}

\begin{center}
\begin{tikzpicture}[scale=1.5]
\coordinate (V) at (0,1);
\coordinate (X) at (2,0);
\coordinate (Xl) at (1.6,0);
\coordinate (Xlc) at (1.8,0);
\coordinate (Xr) at (2.4,0);
\coordinate (Xrc) at (2.2,0);
\coordinate (Y) at (2,2);
\coordinate (Yl) at (1.6,2);
\coordinate (Ylc) at (1.8,2);
\coordinate (Yr) at (2.4,2);
\coordinate (Yrc) at (2.2,2);

\begin{scope}[thick,decoration={
    markings,
    mark=at position 0.5 with {\arrow{>}}}
    ] 
    \only<-4>{\draw[thick, postaction={decorate}] (V) -- (Xl);}
    \only<-4>{\draw[thick, postaction={decorate}] (V) -- (X);}
    \only<-4>{\draw[thick, postaction={decorate}] (V) -- (Xr);}
    \only<-4>{\draw[thick, postaction={decorate}] (Yl) -- (V);}
    \only<-4>{\draw[thick, postaction={decorate}] (Y) -- (V);}
    \only<-4>{\draw[thick, postaction={decorate}] (Yr) -- (V);}
    \draw[thick, postaction={decorate}] (Ylc) -- (Xlc);
    \draw[thick, postaction={decorate}] (Y) -- (X);
    \draw[thick, postaction={decorate}] (Yrc) -- (Xrc);
\end{scope}
\only<-4>{\draw[thick, fill=white] (V) circle (0.1) node {v};}
\draw[thick, fill=white] (X) circle (0.5) node {X};
\only<-5>{\draw[thick, fill=white] (Y) circle (0.5) node {Y};}
\end{tikzpicture}
\end{center}

\end{frame}

\begin{frame}
\frametitle{Kernel mérete}
\begin{itemize}
\item Alkalmazzuk ezeket a szabályokat amíg lehet.
\pause
\item Amikor már nem lehet:
\begin{itemize}
\item 1. szabályt nem lehet, mert: minden él legfeljebb k háromszög része.
\item 2. szabály nem lehet, mert: minden csúcs része valamely háromszögnek.
\pause
\end{itemize}
\begin{footnotesize}
\item Tfh. van k méretű feedback arc set: F.\pause
\item Ennek egy élére: 2 végpont + k db csúccsal lehet háromszögben (1. szabály nem alkalmazható)\pause
\item Minden csúcs benne van egy háromszögben (2. szabály nem alkalmazható)\pause
\item Minden háromszögben van él F-ből. \pause
\item Leszámláltunk k db F-beli élre darabonként k+2 csúcsot, kihagyás nélkül.\pause
\item A gráfban legfeljebb $k(k+2)$ csúcs van, ha megoldható.
\end{footnotesize}
\end{itemize}
\end{frame}

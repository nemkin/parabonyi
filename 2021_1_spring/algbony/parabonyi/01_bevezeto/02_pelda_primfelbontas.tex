\begin{frame}[t]
\frametitle{Példa: Prímtényezős felbontás}

Feladat: számok prímtényezős felbontását megadni.

\mkspace
\begin{columns}
\begin{column}{0.487\textwidth}
$4503599627370496 = 2^{52}$
\end{column}
\begin{column}{0.487\textwidth}
$1125897758834689 = 524287 \cdot 2147483647$
\end{column}
\end{columns}
\mkspace

\begin{itemize}
\item Input mérete: 16 számjegy.
\item Kézzel melyiket fogjuk tudni hamarabb megadni?
\item Számítógép: sokkal több számjegyre hasonlóan (pl. csak 10-nél kisebb prímek vannak benne $\leftrightarrow$ RSA kódolás).
\end{itemize}

\note{Ugyanolyan sok számjegyből állnak a számok, tehát ugyanolyan hosszú az input méretünk, mégis az elsőt
nagyon gyorsan meg lehet találni, a másodikat sokkal lassabban.}

\end{frame}


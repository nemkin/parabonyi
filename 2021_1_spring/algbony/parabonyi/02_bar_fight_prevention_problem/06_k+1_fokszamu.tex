\begin{frame}
\frametitle{k+1}
\begin{itemize}
\item Aki k-nál nagyobb fokszámú azt nem engedhetem be, mert akkor a szomszédjait kellene kitiltani, akik k-nál többen vannak.
\item Ha valakit kitiltok akkor k-t csökkentem eggyel.
\item Maradék gráf: 1...k fokú csúcsok. Minden kitiltás így k vagy kevesebb konfliktust fog megoldani a továbbiakban.
\item Ha több mint $k^2$ élünk van akkor biztosan nem megoldható a feladat, készen vagyunk.
\item Ha $k^2$ vagy kevesebb élünk van, akkor legfeljebb $2k^2$ csúcsunk lehet (minden élnek két vége van és nincs 0 fokú csúcs).
\item ${2k^2 \choose k}$ mostmár $k\leq{}10$-re már jobb mint az előbbi $2^n$.
\end{itemize}
\end{frame}

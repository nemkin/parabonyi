\begin{frame}
\frametitle{Feladat}
Képzeljük el, hogy biztonsági őrként dolgozunk egy falusi bárban. Péntekenként nagy tömeg szokott lenni és
általában bunyóban végződik a történet... A mi feladatunk kidobni az ittas vendégeket, ami nagyon fárasztó
és nem túl mókás. Elhatározzuk, hogy megelőző intézkedéseket teszünk...

Mindenkit ismerünk a faluban és azt is tudjuk ki kivel nincs jóban, kik fognak várhatóan összeverekedni.
A tervünk tehát az, hogy csak olyan embereket engedünk be a bárba, akik jóban vannak egymással, így
elkerüljük a verekedést.

Azonban a bár menedzsmentje maximalizálni akarja a profitot, ezért azt a kikötést teszi, hogy legfeljebb
$k$ darab vendéget lehet elutasítani az ajtóban.

A feladat tehát a következő: Ismerjük a bárba bejövő emberek listáját ($n$ ember), minden emberpárra
tudjuk, hogy fognak-e verekedni ha mindkettőjüket beengedjük. Ki kell találni, hogy be lehet-e úgy
engedni az embereket, hogy legfeljebb k darab embert utasítunk el, úgy hogy bent ne törjön ki verekedés.
\end{frame}
